\chapter{Evaluation} \label{cha:evaluation}

In diesem Kapitel wird die Containerisierung von JULEA evaluiert. Dabei werden die Testumgebung und die verwendeten Benchmarks vorgestellt sowie die gemessenen Werte diskutiert.

\section{Benchmarkumgebung}

Eines der möglichen Variablen, welches das Benchmarkergebnis substantiell beeinflussen kann, ist die Benchmarkumgebung. Die Benchmarkumgebung lässt sich als drei verschiedene Komponenten betrachten. Das eine ist die Hardware, auf welcher der Benchmark ausgeführt wird. Das zweite ist das unterliegende Betriebssystem und dessen Komponenten auf welchen der Benchmark ausgeführt wird und das dritte sind die anderen Prozesse, welche sich auf dem System Ressourcen mit den Benchmarkprozesse(n) teilen. Bevor detaillierter auf den eigentlichen Benchmark eingegangen wird, werden die drei Komponenten der Benchmarkumgebung genauer betrachtet.

\subsection{Hardware}

Der Benchmark wurde auf einem HPC-Cluster der Otto-von-Guericke-Universität Magdeburg ausgeführt. Das Cluster selber besteht aus mehreren "Knoten" (engl. Nodes), welche wie eigenständige Computer agieren. Es gibt Knoten mit verschiedenen Hardware-Spezifikationen. Der Benchmark wurde stets auf einem Knoten mit folgender Hardware-Spezifikation ausgeführt:

\begin{itemize}
    \item CPU: 1x AMD Epyc 7443 @ $2,85\text{GHz}$
    \item RAM: 128GB DDR4
    \item /home, /data, /opt/spack: CephFS
    \item /\*\* (auch /tmp): $1\text{TB}$ SATA-SSD  
\end{itemize}

In beiden Fällen liegt das Kompilat im /home-Verzeichnis, währenddessen die Speicherorte von JULEA unter /tmp liegen. Insbesondere ist zu beachten, dass für Apptainer das /tmp-Verzeichnis des "Host"-Systems mit einem Bind-Mount in den Container eingebunden wurde. Somit ist die Umgebung beider Ausführungsarten so weit wie möglich identisch.

Zusammenfassend lässt sich sagen, dass die Hardware in beiden Fällen identisch ist und es somit zu keiner Verzerrung der Benchmarkergebnisse durch unterschiedliche Hardwarekonfigurationen geben sollte. 

\subsection{Betriebssystem und Betriebssystemkomponenten}

Das Betriebssystem, auf welchem der Benchmark ausgeführt wurde, ist CentOS 8. Die Betriebssystemversion im Container ist im gegensatz dazu Ubuntu 24.04. Dieser Unterschied ist allerdings nicht relevant. Der größte Unterschied zwischen den beiden Betriebssystemen würden die unterschiedlichen Library-Versionen sein. Die unterschiedlichen Library-Versionen spielen allerdings keine Rolle, da hier die Libraries mit Spack (gcc) kompiliert wurden. Somit gibt es hierbei auch keine Unterschiede. Es wurde in beiden fällen GCC 12 verwendet. Unterschiede in der Performance durch einen deutlich neueren Kompilierer sind somit auch ausgeschlossen. In beiden fällen wird der CentOS 8 Linux-Kernel des Host-Systems verwendet, da in einer Container-Umgebung kein separater Kernel gebootet wird.

Hier sind die beiden Umgebungen somit auch in den relevanten Komponenten identisch. Somit sollte hier auch keine Verzerrungen der Benchmarkergebnisse durch unterschiedliche Betriebssystem-Konfigurationen geben.

\subsection{Andere Prozesse}

Der Benchmark musste sich die Umgebung nur mit dem Betriebssystem und dessen Prozessen teilen. Somit ist es unwahrscheinlich, dass das System durch andere Prozesse stark beeinflusst wurde und es dadurch zu Milderungen der Benchmarkergebnissen kam. Der Benchmark-Prozess muss sich mit keinen anderen Programmen anderer Nutzer die Ressourcen teilen, da der Benchmark auf einem dedizierten Knoten ausgeführt wurde. Näheres dazu wird in \cref{sec:benchmark-execution} erläutert.

Hier gibt es keine Unterschiede zwischen den beiden Ausführungsarten, und somit sollten sich die Benchmarkergebnisse nicht durch andere Prozesse auf dem System unterschiedlich verhalten.

Abschließend lässt sich sagen, dass die Benchmarkergebnisse sehr wahrscheinlich nicht durch unterschiedliche Benchmarkumgebungen verzerrt werden, da die Benchmarkumgebungen in den relevanten Komponenten (Hardware, Betriebssystem, Betriebssystemkomponenten und andere Prozesse) identisch sind.   

\section{Ausführung des Benchmarks} \label{sec:benchmark-execution}

Für die Ausführung des Benchmarks auf dem HPC-Cluster müssen die Ressourcen allokiert und die Ausführung der Programme geschedult werden. Auf dem HPC-Cluster der OvGU wird das über das SLURM-System realisiert. Die ausführlichen SLURM-Skripte sind im Anhang zu finden (siehe \cref{cha:appendix-benchmarkskripts}).

Für das SLURM-Skript wurden folgende Parameter gesetzt

\begin{itemize}
    \item --constraint=epyc3: Begrenzt die Allokation von Nodes auf den oben genannten CPU
    \item --exclusive: Node wird exklusiv für den Benchmark reserviert
    \item --time=0-2:30:00: Maximale ausführungsdauer von $2,5\text{h}$
    \item --mem=16GB: 16GB RAM auf dem Node werden reserviert
    \item --array=1-10: zehnfache Ausführung des Benchmarks
    \item --output=./res/apptainer/benchmark-\%a.out: Datei in welches das STDOUT geschrieben wird
    \item --error=./res/apptainer/benchmark-\%a.err: Datei in welches das STDERR geschrieben wird 
\end{itemize}

\section{Erörterung der Benchmarkmetriken}

Um genauer zu verstehen, welche Faktoren einen Einfluss auf die Benchmarkergebnisse haben, ist es wichtig die Bedeutung der Benchmarkmetriken, sowie deren unterliegende Technologien zu verstehen. 
Die hier verwendeten Benchmarks sind in JULEA integriert und in verschiedene Kategorien unterteilt.

Diese Kategorien (Benchmarkmetriken) sind stets in einem spezifischen Format angegeben, dieses Format ist einem POSIX-Pfad ähnlich. Die Benchmarkmetriken sind in verschiedene (Unter-)Kategorien unterteilt. Die Struktur einer Benchmarkmetrik ist somit in folgender Form: /Kategorie/Unterkategorie-0/../Unterkategorie-n-1/Unterkategorie-n.

JULEAs Benchmarkergebnisse liefern einige Kennzahlen. Die aussagekräftigste Kennzahl, um die Performance zweier Systeme zu vergleichen ist die Anzahl der Operationen pro Sekunde (operations). Eine Operation ist eine Ausführung der beschriebenen Benchmarkmetrik. So ist z. B. "/kv/get" eine "get" anfrage an das KV-Backend von JULEA. 

Nachfolgend werden die Benchmarkmetriken in Kategorien unterteilt und erläutert.

\subsection{Objektspeicher}

JULEAs Objektspeicher-Backends werden durch Benchmarkmetriken mit dem Präfix "/object" getestet.

Dabei gibt es eine Differenzierung zwischen "/object/object" und "/object/distributed-object". "/object/object" repräsentiert die Operationen auf Objekten, welche an einem einzelnen Dateisystem gespeichert sind. "/object/distributed-object" repräsentiert die Operationen auf Objekten, welche verteilt gespeichert sind \cite[Vgl. S. 718]{kuhnJULEAFlexibleStorage2017}

\subsection{Key-Value}

Die Benchmarks für die Key-Value-Komponente in JULEA haben das Präfix "/kv".

Hierbei werden alle üblichen Operationen einer Key-Value-Datenbank gebenchmarkt. Dazu gehören das Einfügen, Aktualisieren, Löschen, sowie Lesen von Key-Value-Paaren \cite[Vgl. S. 718]{kuhnJULEAFlexibleStorage2017}.

\subsection{Item-Store}

Der Item-Store – alle Benchmarkmetriken, welche mit "/item" beginnen – repräsentieren die von JULEA unterstützten Operationen auf dem Item-Store. Dieser soll ein cloud-ähnliches Interface bereitstellen und nutzt den Key-Value-Store sowie den Objektspeicher um dies zu erzielen \cite[Vgl. 718]{kuhnJULEAFlexibleStorage2017}.

\subsection{Datenbank}

Die Datenbank-Benchmarkmetriken haben "/db" als Präfix. Hier werden übliche Datenbankoperationen wie das Einfügen, Löschen, Aktualisieren von Einträgen, sowie das Erstellen und Löschen von Schemas gebenchmarkt \cite[Vgl. S. 718]{kuhnJULEAFlexibleStorage2017}.

\subsection{Ausgeschlossene Benchmarkmetriken}

Die Benchmarkmetriken Message ("/message"), Cache ("/cache") und Background Operation ("/background") wurden in dieser Auswertung nicht berücksichtigt. Die Benchmarkmetriken Message und Background wiesen eine sehr hohe Varianz auf, was eine Aussage mit hoher Sicherheit nicht möglich macht. Die Cache-Benchmarkmetrik ist für den Vergleich beider Lösungsansätze nicht relevant, da Apptainer keine Virtualisierung des System-Memories vornimmt. Es wird transparent auf den Zwischenspeicher des Hostsystems zugegriffen.   

\section{Vor-Auswertung der Benchmarkergenisse}

Um die Benchmarkergebnisse effektiv auswerten zu können, ist im ersten Schritt wichtig, die Verteilung der Messwerte zu betrachten. Ein Boxplot ist eine gängige Methode, Verteilungen darzustellen \cite[Vgl. S. 1]{majawExploringDataDistributions2023}. Nachfolgend werden die Verteilungen der nativen und containerisierten Ergebnisse beispielhaft betrachtet. Auf eine vollständige Darstellung aller Verteilungen wird verzichtet, da sie sich weitgehend ähneln.

Die gesamten Benchmarkergebnisse sind im Anhang in \cref{cha:appendix-benchmarkdaten} verzeichnet.

\pagebreak

\subsection{Native}

Die Verteilungen der Messwerte des Benchmarks, welche Nativ auf dem Host-System ausgeführt wurden, lassen sich aus der Grafik in \cref{fig:boxplot_object} entnehmen.

\begin{figure}[H]
    \includesvg[width=1\linewidth]{benchmark/vis/compressed/boxplots/system/object/boxplot.svg}
    \caption{Verteilung der Benchmarkergebnisse für den Objektspeicher (Nativ)}
    \label{fig:boxplot_object}
\end{figure}
\FloatBarrier

Es ist erkenntlich, dass die Verteilungen der Messwerte oftmals verzerrt sind, und es einige Ausreißer gibt. Bei dieser Benchmarkbetrachtung soll keine genauere Betrachtung der Extremwerte stattfinden. Darum ist die Betrachtung des Medianwertes als statistisches Mittel hier Sinnvoller \cite[Vgl. S. 15f]{stengelStatistikUndAufbereitung2011}. 

Des Weiteren liegen die gemittelten Messwerte der einzelnen Benchmarkmetriken zum Teil weit auseinander, was zur Folge hat, dass der grafische Vergleich zwischen den verschiedenen Ausführungsarten des Benchmarks mit absoluten Werten nicht sehr aussagekräftig ist. Darum wird der Vergleich der verschiedenen Ausführungsarten des Benchmarks mithilfe des "Speedups" berechnet. Dieser betrachtet die relative Verbesserung zwischen zwei Messwerten. Die Formel für den Speedup kann man \cref{eq:speedup} entnehmen.


\begin{equation} \label{eq:speedup}
S_{\frac{\text{ops}}{\text{s}}} = \frac{O_{\text{containerized}}}{O_{\text{native}}}
\end{equation}


\subsection{Containerisiert}

Die Schlüsse, welche bei der Betrachtung der nativen Ausführung des Benchmarks gezogen wurden, lassen sich auch auf die containerisierte Ausführung des Benchmarks übertragen. Die Messergebnisse des Benchmarks, welche in einem Apptainer-Container ausgeführt wurden, lassen sich aus der Grafik in \cref{fig:boxplot_object_apptainer} entnehmen.


\begin{figure}[H]
    \includesvg[width=1\linewidth]{benchmark/vis/compressed/boxplots/apptainer/object/boxplot.svg}
    \caption{Verteilung der Benchmarkergebnisse für den Objektspeicher (Apptainer)}
    \label{fig:boxplot_object_apptainer}
\end{figure}


Es ist bereits bei Betrachtung beider Boxplots ersichtlich, dass beide Ergebnisse größtenteils ähnlich sind. Um noch genauer die Performance-Unterschiede zu erkennen, wird im nächsten Schritt der Speedup für die einzelnen Benchmarkmetriken betrachtet.


\pagebreak

\section{Vergleich der Benchmarkergebnisse}

Um einen besseren Überblick über den Unterschied zwischen beiden Benchmarkergebnissen zu erhalten, wird der errechnete Speedup um -1 verschoben. Das hat zur Folge, dass anstelle vom Wert 1, nun der Wert 0 identische Performance bedeutet. 
Negative Werte bedeuten, dass die containerisierte Ausführung des Benchmarks schlechter ist als die native Ausführung. Positive Werte bedeuten, dass die containerisierte Ausführung des Benchmarks besser ist als die native Ausführung. 

Die übergeordneten Kategorien der Benchmarkmetriken werden nachfolgend einzeln betrachtet.

\subsection{Objektspeicher (POSIX)} \label{sec:object-speedup}

\begin{figure}[H]
    \centering
    \includesvg[width=1\linewidth]{benchmark/vis/compressed/differences/object/difference_operations.svg}
    \caption{Speedup der Benchmarkergebnisse für den Objektspeicher}
    \label{fig:speedup_object}
\end{figure}

\FloatBarrier

Als Erstes wird der Speedup der Benchmarkmetrik "object" betrachtet. Hier ist zu erkennen, dass die containerisierte Ausführung des Benchmarks stets zwischen $1\%$ und $3\%$ langsamer ist. Die Benchmarkmetrik "object" misst dabei die Performance der Objektverwaltung. Diese Objektverwaltung läuft bei diesem Benchmark über das Dateisystem ("POSIX"-Backend) \cite[Vgl. S. 719]{kuhnJULEAFlexibleStorage2017}. Somit ist es naheliegend, dass in diesem Fall das Dateisystem ein Bottleneck sein könnte. Allerdings ist diese sehr konsistente Performance-Verschlechterung im Vergleich zur nativen Ausführung auf den ersten Blick nicht erklärbar, da beide Ausführungsweisen auf das gleiche Verzeichnis im Host-Dateisystem zugegriffen haben (/tmp). 

Im Fall von Apptainer wurde das Verzeichnis mithilfe eines Bind-Mounts in den Container eingebunden. Für die Containertechnologie Docker gab es bereits eine Performance-Analyse von Bind-Mounts, und es konnten keine signifikanten Performance-Einbußen festgestellt werden \cite[Vgl. S. 4f]{dordevicFileSystemPerformance2022}. 

Somit lässt sich darauf schließen, dass die Bind-Mounts nicht die Ursache für die Performance-Einbußen sind. Bei Apptainer besteht jedoch noch die Besonderheit, dass die Container-Images ein SquashFS-Dateisystem verwenden. Dieses Dateisystem wird beim Start des Containers in das Host-Dateisystem eingehängt. 

Hier gibt es bei Apptainer zwei Möglichkeiten, wie dieses Container-Image eingehängt wird: \\
Entweder wird das Image mit dem Kernel-SquashFS-Treiber eingehängt, was privilegierte Rechte benötigt, oder es wird mithilfe des SquashFS-FUSE-Treibers eingehängt. Hierfür werden keine privilegierten Rechte benötigt. Allerdings ist laut eigenen Aussagen von Apptainer die Performance des FUSE-Treibers schlechter als die des Kernel-Treibers. Somit könnte dies eine mögliche Ursache für die Performance-Einbußen sein \cite{apptainerSecurityApptainerApptainer}.  

\subsection{Key-Value (lmdb)} \label{sec:kv-speedup}

\begin{figure}[H]
    \centering
    \includesvg[width=0.7\linewidth]{benchmark/vis/compressed/differences/kv/difference_operations.svg}
    \caption{Speedup der Benchmarkergebnisse für Key-Value}
    \label{fig:speedup_kv}
\end{figure}

\FloatBarrier

Die Key-Value-Komponente von JULEA ("kv") hingegen weist kaum eine Veränderung der Performance zwischen der nativen und containerisierten Ausführung auf. Die Performance schwankt innerhalb von $\pm 1\%$, und es kann nicht ausgeschlossen werden, dass dies – aufgrund der geringen Abweichung – nur eine Messungenauigkeit ist. Somit ist es wahrscheinlich, dass die Performance der Key-Value-Komponente von JULEA nicht durch die Containerisierung beeinflusst wird.

\subsection{Item-Store}

\begin{figure}[H]
    \centering
    \includesvg[width=1\linewidth]{benchmark/vis/compressed/differences/item/difference_operations.svg}
    \caption{Speedup der Benchmarkergebnisse für den Item-Store}
    \label{fig:speedup_item}
\end{figure}

\FloatBarrier

Der Item-Store ist im Vergleich zu den vorherigen Benchmarkmetriken anders. Er baut auf den Implementierungen des Objektspeichers (\cref{sec:object-speedup}) und des Key-Value-Stores (\cref{sec:kv-speedup}) auf.

\subsubsection{Item}

Zum einen wird für die Operationen "/item/\{item, collection\}/create", sowie "/item/\{item, collection\}/delete" auf die Mechanismen "/kv/put", "/kv/delete" aufgebaut. Das begründet den insignifikanten Speedup für die Operationen "/item/item/create" und "/item/\{item, collection\}/delete". 

Die Operationen "/item/item/read" und "/item/item/write" bauen hingegen auf den Objekt-Mechanismen "/object/distributed-object/read" und "/object/distributed-object/write" auf.

Bei den Benchmarkmetriken "/item/item/read" und "/item/item/write" ist – wie zu erwarten – auch ein Performance-Verlust wie bei den object-read/write Ergebnissen – zu erkennen. 

Bei "/item/item/read" ist der Performance-Verlust ca. $0.5\%$. Das ist ein um etwa $1,2\%$ höherer Speedup im Vergleich zu ca. $1,7\%$ Performance-Verlust bei "/object/distributed-object/read". 

Bei "/item/item/write" ist der Performance-Verlust etwa $2\%$. Das ist etwas höher als bei "/object/distributed-object/write", welcher ein Performance-Verlust von ca. $1,5\%$ aufweist. Da es sich hierbei um geringe Unterschiede handelt, ist davon auszugehen, dass dies Messungenauigkeiten sind. 

Die Metriken "/item/item/delete" und "/item/collection/delete" haben einen Speedup, welcher unterhalb von $0,3\%$ liegt. Es wird hierbei davon ausgegangen, dass dies Messungenauigkeiten sind.

Wesentlich auffälliger ist hingegen die Metrik "/item/collection/create". Diese hat Performance-Zunahmen von ca. $1,5\%$. Um dieses Ergebnis besser zu verstehen werden nachstehend die Messwertverteilungen der beiden Benchmarkergenisse analysiert. 

\begin{figure}[H]
    \centering
    \includesvg[width=1\linewidth]{benchmark/vis/compressed/differences/comparisons/run_to_run_distribution__item_collection_create.svg}
    \caption{Speedup der Benchmarkergebnisse für den Item-Store}
    \label{fig:speedup_item_collection_create}
\end{figure}

\FloatBarrier

Es ist ersichtlich, dass die Benchmarkergebnisse der nativen Ausführung deutlich konsistenter sind als die der containerisierten Ausführung. Mehr als die Hälfte der nativen Ergebnisse liegen zwischen 5200 und 5300 Operationen/s. Allerdings ist die Streuung der nativen Ergebnisse deutlich größer. 

Die containerisierten Ergebnisse sind, wie bereits angemerkt, deutlich inkonsistenter, jedoch ist die Streuung hier geringer. Die Verteilung der Messwerte ist – im Vergleich zur nativen Ausführung – in die positive Richtung verzerrt. 

Trotz dieser Unterschiede zeigt sich, wie ähnlich die beiden Benchmarkergebnisse im Kern sind. Während bei der nativen Ausführung die meisten Ergebnisse zwischen 5200 und 5300 Operationen/s liegen, befinden sich die meisten containerisierten Ergebnisse zwischen 5200 und ca. 5325 Operationen/s. Angesichts der insgesamt 10 Iterationen des Benchmarks pro Ausführungsweise ist dies nicht signifikant genug, um von einer relevanten Performance-Änderung zu sprechen. Es handelt sich hierbei sehr wahrscheinlich um Messungenauigkeiten. 


\subsection{Datenbank (SQLite)}

\begin{figure}[H]
    \centering
    \includesvg[width=1\linewidth]{benchmark/vis/compressed/differences/db/difference_operations.svg}
    \caption{Speedup der Benchmarkergebnisse für Datenbanken}
    \label{fig:speedup_db}
\end{figure}

\FloatBarrier

Die Speedups der Datenbank-Benchmarkergebnisse – welche aus \cref{fig:speedup_db} entnommen werden können – für die Metriken "/db/entry/delete", "/db/entry/insert" und "/db/iterator/get-simple" liegen deutlich innerhalb von $\pm 1\%$. Die Werte sind zu gering, um von einer signifikanten Performance-Änderung zu sprechen. Es wird daher davon ausgegangen, dass es keine signifikanten Performance-Änderungen bei den genannten Benchmarkmetriken gibt.

Die Metrik "/db/entry/update" hingegen ist beinahe $3\%$ performanter. Um hier eine genauere Aussage zu treffen, wird im nächsten Schritt die Verteilung der Messwerte der Benchmarkergebnisse analysiert.

\begin{figure}[H]
    \centering
    \includesvg[width=1\linewidth]{benchmark/vis/compressed/differences/comparisons/run_to_run_distribution__db_entry_update.svg}
    \caption{Verteilung der Benchmarkergebnisse für "/db/entry/update"}
    \label{fig:mdist_db_entry_update}
\end{figure}

\FloatBarrier

Aus \cref{fig:mdist_db_entry_update} ist ersichtlich, dass die containerisierte Ausführung wesentlich konsistentere Ergebnisse erzieht hat. 
Die nativen Ergebnisse sind etwas inkonsistenter und haben einen Ausreißer nach unten. 

Allerdings sieht man auch hier, dass sich die Ergebnisse von der containerisierten und nativen Lösung zwischen ca. 5100 und 5250 Operationen/s mehrheitlich befinden. 

Dies ist ein weiteres Beispiel dafür, wie ähnlich die beiden Benchmarkergebnisse sind. Die Messwerte liegen so nah beieinander, dass man in Anbetracht der geringen Messwertanzahl keinen signifikanten Speedup feststellen kann.

\begin{figure}[H]
    \centering
    \includesvg[width=1\linewidth]{benchmark/vis/compressed/differences/comparisons/run_to_run_distribution__db_schema_create.svg}
    \caption{Verteilung der Benchmarkergebnisse für "/db/schema/create"}
    \label{fig:mdist_db_schema_create}
\end{figure}

\begin{figure}[H]
    \centering
    \includesvg[width=1\linewidth]{benchmark/vis/compressed/differences/comparisons/run_to_run_distribution__db_schema_delete.svg}
    \caption{Verteilung der Benchmarkergebnisse für "/db/schema/delete"}
    \label{fig:mdist_db_schema_delete}
\end{figure}

Für "/db/schema\{create, delete\}" ist in \cref{fig:mdist_db_schema_create} und \cref{fig:mdist_db_schema_delete} ebenfalls ersichtlich, dass bei Betrachtung der Messwertverteilung kein signifikanter Speedup festgestellt werden kann. Die Ergebnisse liegen so nah beieinander, dass nicht ausgeschlossen werden kann, dass es sich um Messungenauigkeiten handelt.

\FloatBarrier

\subsection{Fazit der Benchmarkergebnisse}

Insgesamt lässt sich ein deutlicher Trend erkennen. Die Benchmarkergebnisse unterscheiden sich nur sehr geringfügig. Selbst da wo es zu Performance-Verlusten kommt, sind diese sehr gering und liegen unterhalb von $3\%$. \
Viele der Benchmarkergebnisse sind effektiv – unter Berücksichtigung der Messungenauigkeiten – identisch.

Dies ist kein überraschendes Ergebnis, es wurde bereits in vielen anderen Publikationen ein ähnliches Verhalten wie hier festgestellt \cite[Vgl. S. 2589ff]{huExploringPerformanceSingularity2019}

Sollte man starken Gebrauch von Dateisystemfunktionen machen, so kann es zu Performance-Einbußen kommen. Allerdings ist in vielen Fällen ein Performance-Verlust von unter $3\%$ akzeptabel.