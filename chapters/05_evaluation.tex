\chapter{Evaluation}

\section{Native vs Containerized vs Virtualized}

Der Performance-Test wird auf einer Node eines Compute-Custers ausgeführt. Die Node hat folgende Spezifikationen:

\begin{itemize}
    \item CPU: AMD Epyc 7443, 2,85 GHz, 24 Core (48 Threads)
    \item RAM: 16 GB
    \item Speicher: NFS?
\end{itemize}

Für den Benchmark wird ein bereits existierendes Skript "benchmark.sh" verwendet, welches die Performance von allen möglichen Operationen, welche JULEA hat, getestet. Um das benchmark möglichst von externen Faktoren abhängig zu machen, wird der Benchmark auf einer Node eines Clusters als einziges Programm ausgeführt (exklusiv). Das benchmark wird 10-mal nacheinander ausgeführt, um die Konfidenz der Messergebnisse zu erhöhen (Vgl. \cite{kaliberaRigorousBenchmarkingReasonable2013} S. 11ff). 

\subsection{Native}



\subsection{Containerized}

\subsection{Virtualized}

\todo[inline]{Wie virtualisiert man auf dem Cluster?}




