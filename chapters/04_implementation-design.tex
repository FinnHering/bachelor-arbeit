\chapter{Implementierung und Design}
\label{cha:implementation_design}

\textit{In diesem Kapitel wird die Implementierung und das Design der Containerisierung von JULEA erläutert. Zu Beginn wird noch einmal kurz auf den Ist-Zusatand eingegangen.}

\section{Ist-Zustand}

Eine Julea-Entwicklungsumgebung kann aktuell in zweierlei Formen aufgesetzt werden:

Zum einen kann man die nötigen Abhängigkeiten manuell über das Betriebssystem installieren. Dies bring den vorteil mit sich, dass diese automatisiert korrekt installiert werden und man diese üblicherweise auch bereits fertig kompiliert bereitgestellt bekommt. Der Nachteil ist, dass man nur ein begrenztes spektrum an Versionen zur Verfügung hat und man nicht entscheiden kann, welche Features bei der Abhängigkeit aktiviert oder deaktiviert seien sollen. Des Weiteren ist es durchaus möglich, dass die benötigte Abhängigkeiten nicht in der Paketrepository des Betriebssystems vorhanden sind. 

Zum anderen kann man die Abhängigkeiten über ein mitgeliefertes Skript installieren ("install-dependencies.sh"). Dieses Skript nutzt den Paketmanager "spack" um die Abhängigkeiten zu installieren. Dadurch ist man nicht mehr auf das Paketrepository des Betriebssystems angewiesen und könnte auch spezifische Versionen der Abhängigkeiten installieren. Der Nachteil ist, dass diese Abhängigkeizen kompiliert werden müssen. Dies benötigt Zeit und Rechenleistung. 

Julea wird bereits mit Github-Actions automatisiert kompiliert und getestet. Es gibt bereits eine Pipeline, welche Ubuntu-Container mit den benötigten Abhängigkeiten für Julea erstellt. Momentan finden dies Container keine Verwendung, in den nachgelagerten CI-Pipelines. 

\todo[inline]{Vergleich von Compilezeiten zwischen unterschiedlichen maschinen}

Um die Entwicklung von Julea zu erleichtern wäre es sinnvoll eine Betriebssystemagnostische Entwicklungsumgebung zu haben, wo die Abhängigkeiten bereits vorinstalliert sind. Nachfolgend wird erleutert wie dies umgesetzt wurde und welche Containervisualisierungssoftware dafür verwendet wurde.

\section{Benötigte Containerimages}

Um die Containerimages möglichst effizient zu gestalten, ist es sinnvoller mehrere domainspezifische Containerimages zu erstellen, als ein großes Containerimage, welches dann ggf. Komponenten beinhaltet, welche man nicht benötigt.

Bei betrachtung der Julea-Repository fallen zwei Domainen auf. Zu einem ist es eine Entwicklungsumgebung, welche alle optionalen und nicht-optionalen Abhängigkeiten von Julea zum Kompilieren benötigt, sowie eine Testumgebung, um Julea zu testen. Außerdem sollte man eine Deploymentumgebung haben, worin eine bereits kompilierte version Julea mit den benötigten laufzeitabhängigkeiten enthalten ist.

Nun könnte man für jede Domain ein eigenes Containerimage erstellen. Dies ist allerdings im Falle von der Testumgebung und der Entwicklungsumgebung nicht unbedingt notwendig, da es durchaus üblich ist, dass man während der Entwicklung auch tests schreibt und diese ausführt. Aus diesem Grund und um nicht zu viele verschiedene Versionen von Containerimages zu haben, wird für die Entwicklungsumgebung und die Testumgebung ein Containerimage erstellt.

Außerdem wird Julea gegen verschiedene Kompiler und Betriebssysteme kompiliert und getestet. Somit muss jeder Entwicklungs- und Produktivcontainer in jeder Kombination von Kompiler und Betriebssystem vorhanden sein. 

Aktuell wird Julea mit CLang und GCC kompiliert und auf Ubuntu 20.04, Ubuntu 22.04 und Ubuntu 24.04 kompiliert und getestet.

Außerdem gibt es 2 möglichkeiten um die benötigten Abhängigkeiten, welche Julea zum Kompilierungszeitpunkt benötigt, zu installieren. Entweder nutzt man das mitgelieferte Skript "install-dependencies.sh", oder man stellt die Abhängigkeiten über das Betriebssystem bereit. Somit muss der Entwicklungs- sowie der Produktivcontainer in jeder Kombination von Betriebssystem, Kompiler und Abhängigkeits-Installationsmethode vorhanden sein.

Grundlegend gibt es somit 2 Containerimages: julea-dev und julea-prod. Beide diese Containerimages haben jeweils 12 varianten.

Das Namensschema der Containerimages ist wie folgt: \\
julea-\{dev, prod\}:\{kompiler\}-\{abhängigkeitsquelle\}-\{betriebssystem(version)\}
Somit gibt es folgende Containerimages:

\begin{multicols}{2}
    \begin{itemize}
        \item julea-dev-gcc-system-ubuntu-20.04  
        \item julea-prod-gcc-system-ubuntu-20.04  
        \item julea-dev-gcc-system-ubuntu-22.04  
        \item julea-prod-gcc-system-ubuntu-22.04  
        \item julea-dev-gcc-system-ubuntu-24.04  
        \item julea-prod-gcc-system-ubuntu-24.04  
        \item julea-dev-clang-system-ubuntu-20.04
        \item julea-prod-clang-system-ubuntu-20.04
        \item julea-dev-clang-system-ubuntu-22.04
        \item julea-prod-clang-system-ubuntu-22.04
        \item julea-dev-clang-system-ubuntu-24.04
        \item julea-prod-clang-system-ubuntu-24.04
        \item julea-dev-gcc-spack-ubuntu-20.04   
        \item julea-prod-gcc-spack-ubuntu-20.04   
        \item julea-dev-gcc-spack-ubuntu-22.04   
        \item julea-prod-gcc-spack-ubuntu-22.04   
        \item julea-dev-gcc-spack-ubuntu-24.04   
        \item julea-prod-gcc-spack-ubuntu-24.04   
        \item julea-dev-clang-spack-ubuntu-20.04 
        \item julea-prod-clang-spack-ubuntu-20.04 
        \item julea-dev-clang-spack-ubuntu-22.04 
        \item julea-prod-clang-spack-ubuntu-22.04 
        \item julea-dev-clang-spack-ubuntu-24.04 
        \item julea-prod-clang-spack-ubuntu-24.04 
    \end{itemize} 
\end{multicols}

\section{Aufbau der Dockerfiles}

Der de-facto standard um Containerimages zu bauen, ist das Erstellen einer Containerfile (umgansprachlich Dockerfile). Diese werden von verschiedenen OCI-Containervisualisierungssoftwares, wie Docker, Podman, Buildah unterstützt. Ein alternatives Format stellt Apptainer dar, dass sogennante "Apptainer definition file" Dateiformat. Von der benutzung wird abgesehen, da dieses Format nicht sehr weit verbreitet ist und Apptainer die aus der Containerfile herforgehenden OCI-Containerimages unterstützt. Somit erzieht man mit der Containerfile eine hohe Kompatibilität innerhalb der Containerisierungslandschaft, währenddessen man Apptainer unterstützt welche eine weitverbreitete Containerlösung innerhalb des HPC darstellt. 

Der Aufbau der Containerfile ist wie folgt:

\begin{figure}[!htbp]
    \centering
    \includesvg[width=400pt]{./figures/modell-containerfile.drawio.svg}
    \caption{Containerfile Layer-Graf}
\end{figure}

\FloatBarrier

Der Layer-Graf zeigt auf, dass die Pfade von Spack, sowie System sehr früh schon voneinander abweichen. Dadurch ist es sinnvoll zwei Containerfiles für Spack und System zu erstellen, um die beiden Dateien übersichtlicher zu halten. 

\subsection{"System" Containerfile}

Der Layer-Graf für das Containerfile "System" sieht wie folgt aus. 
\begin{figure}[!htbp]
    \centering
    \includesvg[width=400pt]{./figures/modell-system-containerfile.drawio.svg}
    \caption{Containerfile "System" Layer-Graf}
\end{figure}

Die dazugehörigen Dockerfile sieht wie folgt aus: 

\inputminted{dockerfile}{./code-examples/Dockerfile.system}

\subsection{"Spack" Containerfile}
Der Layer-Graf für das Containerfile "Spack" sieht wie folgt aus.
\begin{figure}[!htbp]
    \centering
    \includesvg[width=400pt]{./figures/modell-spack-containerfile.drawio.svg}
    \caption{Containerfile "Spack" Layer-Graf}
\end{figure}

Die dazugehörige Dockerfile hat folgenden aufbau.

\inputminted{dockerfile}{./code-examples/Dockerfile.spack}

Bei beiden Dockerfiles fällt auf, dass apt-get update und apt-get install stets im selben "RUN" befehl ausgeführt werden. Dies gilt nicht nur für apt-get, sondern auch für andere Paketmanager, welche vor der installation vorgelagerte Schritte ähnlich wie apt-get update benötigen. Sollte man die vorgelagerten schritte in einzelne "RUN" befehle ausführen, hätte das für das erste ausführen von docker build keine implikationen, da dort der cache noch nicht befüllt wurde. Bei allen folgenden ausführungen von docker build, kann es nach Anpassung der apt-get install Zeile zu Problemen führen. Dies liegt daran, dass alle vorgelagerten Schritte immernoch im cache liegen, bei dem apt-get install schritt kommt es allerdings zu einem cache-miss und dieser muss neu evaluiert werden. Wenn nun die paketinformationen, welche für apt-get update im layer cache liegen nicht mehr aktuell sind, wird apt-get install fehlschlagen, da es die Pakete, welche beim Mirror angefragt werden nicht mehr gibt. Um dieses Problem zu beheben, sollte man stets apt-get update und apt-get install im selben "RUN" befehl ausführen. Dadurch wird bei jeder anpassung der benötigten Pakete automatisch auch apt-get update ausgeführt und es kann zu keinen veralteten Paketinformationen kommen. 

\subsection{Zusammenspiel Dockerfile -> Images/Tags}

Welche Dockerfile generiert welche Images/Tags?

\section{Docker Bakefile}

Das Erstellen mehrerer Containerimages aus einer Containerfiles ohne einen automatismus, kann sehr aufwendig werden, da man jedes Image einzeln erstellen lassen muss. Insbesondere bei der erstellung einer großen Menge von Containerimages wie es hier der Fall ist, wäre das manuell Erstellen der Containerimages mit einem zu hohen aufwand verbunden. 

Es gibt eine große Anzahl an Build-Automatisierungs-Tools. Ein sehr bekanntes Programm ist GNU-Make. Eine weitere möglichkeit wäre auch ein einfaches Shell-Skript, welche die erstellung automatisiert. Das Erstellen kann außerdem mithilfe von CI-Pipelines verinfacht werden. Github-Actions hat explizite Aktionen, um Containerimages zu erstellen. Mithilfe einer Matrix könnte man somit bei Github-Actions alle Containerimages unkompliziert erstellen. Eine weitere option ist, das Benutzen einer "Bakefile", dies ist ein spezieller dialekt der Hashicorp Configuration Language (HCL). Bakefiles sind ein feature von Docker Buildx, um mehrere Containerimages vordefiniert zu erstellen. Es löst also genau das oben beschriebene Problem in einem standardisierten und einfachen Weg. Außerdem hat Github-Actions auch eine Action für Docker-Bakefiles, somit lässt es sich auch in die existierende CI/CD-Pipeline integrieren. 

Die Docker-Bakefile sieht wie folgt aus: 

\inputminted{dockerfile}{./code-examples/docker-bake.hcl}

\section{CI}

Um das Erstellen der Container noch komfortabler zu machen, werden diese mithilfe einer CI-Pipeline erstellt und automatisch in ein Container-Repository geladen. Da das Julea-Projekt auf Github gehostet wird und bereits Github-Actions als CI-/CD-Lösung verwendet, wird auf diese Pipeline aufgebaut und Github-Actions verwendet.

Hierbei wird die Pipeline um die Funktionalität des Container-Erstellens sowie veröffentlichen erweitert. Des weiteren werden in den nachgelagerten Test-Pipelines auch die erstellten Container verwendet.

Die CI-Pipeline wird außerdem die erstellten OCI-Containerimages in Apptainer-Images umwandeln und diese als artefakte bereitstellen.


