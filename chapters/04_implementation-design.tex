\chapter{Implementierung und Design}
\label{cha:implementation_design}

\textit{In diesem Kapitel wird die Implementierung und das Design der Containerisierung von JULEA erläutert. Zu Beginn wird noch einmal kurz auf den Ist-Zusatand eingegangen.}

\section{Ist-Zustand}

Julea kann als SPACK-Paket ausgerollt werden; Jeder Entwickler muss entweder die nötigen Abhängigkeiten manuell installieren oder mithilfe von bereitgestellen shell-datein kompilieren lassen. Das ist sehr zeitaufwendig und fehleranfällig.

\section{Benötigte Containerimages}

Was für containerimages soll es geben?

\begin{itemize}
    \item Devcontainer
    \item Deploymentcontainer
    \item Test-Container
\end{itemize}

Welche Tags sollen die Containerimages haben?

Was sollen diese Tags beinhalten?

\begin{itemize}
    \item Devcontainer: 
    
    \begin{itemize}
        \item latest
        \item ubuntu-20.04
        \item ubuntu-22.04
        \item ubuntu-24.04
    \end{itemize}
    
    \item Deploymentcontainer: 
    \begin{itemize}
        \item latest
        \item ubuntu-20.04
        \item ubuntu-22.04
        \item ubuntu-24.04
    \end{itemize}
    
    \item Test-Container 
    \begin{itemize}
        \item gcc-ubuntu-20.04
        \item gcc-ubuntu-22.04
        \item gcc-ubuntu-24.04
        \item clang-ubuntu-20.04
        \item clang-ubuntu-22.04
        \item clang-ubuntu-24.04
        \item gcc-spack-ubuntu-20.04
        \item gcc-spack-ubuntu-22.04
        \item gcc-spack-ubuntu-24.04
        \item clang-spack-ubuntu-20.04
        \item clang-spack-ubuntu-22.04
        \item clang-spack-ubuntu-24.04
    \end{itemize}
    

\end{itemize}

\section{Aufbau der Dockerfiles}

2 Dockefiles: Weil das bauen Spack-Container und Julea-Container fast komplett unterschiedlich ist.

\subsection{Ubuntu Dockerfile}

\subsection{Spack Dockerfile}

\subsection{Zusammenspiel Dockerfile -> Images/Tags}
Welche Dockerfile generiert welche Images/Tags?

\section{Docker Bakefile}

Vereinfacht das erstellen von mehreren Docker-Images