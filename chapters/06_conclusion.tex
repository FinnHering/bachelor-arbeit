\chapter{Fazit} \label{cha:conclusion}

In diesem Kapitel werden die Erkenntnisse sowie Ergebnisse der Arbeit zusammengefasst und es wird einen Ausblick auf mögliche zukünftige Arbeiten gegeben. 

\section{Zusammenfassung}

Wie in Kaptiel \cref{cha:evaluation} erwähnt ist die Containerisierung von komplexeren Applikationen oder Applikationen, welche eine lange Zeit benötigen Kompiliert zu werden von vorteil für den Endnutzer der Applikation, da er die Applikation nicht selbst kompilieren muss und mithilfe des Containers auch eine reproduzierbare Umgebung erhält. 

Mithilfe von Dockerfiles, den Docker-Bake-Dateien und Docker Buildx konnte der Erstellprozess der Containerimages standardisiert und auch einfach ausführbar gemacht werden. Die anschließende definition des CI-Workflows hat dann anschließend das Erstellen automatisiert, somit der Entwickler nun nicht mehr die Containerimages manuell erstellen und veröffentlichen muss. 

In der Evaluation konnte des Weiteren kein starker Performanceverlust durch die Containerisierung festgestellt werden. In den meisten fällen waren beide Installationen (Container und native Installation) gleich schnell. In einigen Fällen könnten vereinzelt minimale Performanceverluste festgestellt werden, welche für die meisten Anwendungsfälle wahrscheinlich nicht relevant sind.      

\section{Zukünftige Arbeit}

\subsection{Vertiefung der Benchmarks}

In dieser Arbeit wurde eine erste Evaluation der Performance von Containerimages durchgeführt. Allerdings wurde keine genauere Betrachtung der Konfidenz der Ergebnisse durchgeführt. Die Zehnfache ausführung der Benchmarks erzeugt zwar für den Großteil der Benchmarks aussagekräftige Ergebnisse, allerdings gibt es immernoch einige Ergebnisse wo die Standardabweichung sehr hoch ist, was eine Aussage über die Performance bei diesen Benchmarkmetriken schwierig macht und diese Ergebnisse somit in dieser Arbeit nicht berücksichtigt wurden.

\subsection{Automatisierte Konvertierung von OCI-Images zu Apptainer-Images}

In der Implementierung wurde das Erstellen von OCI-Containerimages automatisiert unter der Begründung, dass Apptainer die Möglichkeit anbietet OCI-Images zu Apptainer-Images zu konvertieren und, dass Docker (und somit auch OCI-Containerimages) der de-facto Standard für Containerisierung ist, somit auch über ein größeres Ökosystem verfügt und somit auch in mehr Umgebungen bereits integriert ist, was die "Developer Experience" verbessert.

Allerdings findet Apptainer eine Nische in der Containerisierung von Applikationen, welche auf HPC-Systemen laufen. JULEA ist ein Beispiel für eine solche Applikation. Somit wäre es Sinnvoll, sich in der Zukunft darüber Gedanken zu machen, wie man OCI-Containerimages automatisiert in Apptainer-Images konvertieren kann (idealerweise mithilfe von CI), um auch das vollautomatisierte Ausrollen von Apptainer-Images auf HPC-Systemen zu ermöglichen (CD).



